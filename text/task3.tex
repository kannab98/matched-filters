\subsection{Задание 3. Согласованная фильтрация линейно-частотно
модулированного сигнала}
В этом задании на примере ЛЧМ сигнала подробно исследуются
особенности фильтрации сложных сигналов. 

Выбрать среднюю частоту заполнения 1000Гц,
длительность ЛЧМ сигнала менять в пределах от 10мс до 100мс
девиацию частоты изменять от 400Гц до 1000Гц 

Пропустить ЛЧМ сигнал через согласованный фильтр. Качественно
проанализировать, чем определяются основные параметры выходного сигнала:
величина его максимума и степень укорочения сигнала, временное положение
максимума. Получить и построить графики следующих зависимостей, оставляя
среднюю частоту неизменной:

Задание 3
\textit{Текст Ильи}

1) Максимум выходного сигнала достигается в момент окончания входного сигнала, соответственно,
чем длинне входной сигнал тем позже наступит пик выходного.  Расстояние между нулем амлпитуды
и ее максимумом не зависит от длительности входного сигнала. 

При изменении девиации частоты входного сигнала не меняется положение максимума амплитуды
во времени, но меняется длительность между пиковым и нулевым значением амплитуды.

2) При увеличении длительности сигнала прямо пропорционально возрастает амплитуда сигнала на выходе.

Изменение девиации частоты на амплитуду не виляет.



Сигнал на выходе согласованного фильтра имеет форму корреляционной функции
полезного сигнала. 

Пиковое значение выходного сигнала согласованного фильтра достигается не раньше, чем окончится
импульсный сигнал, поступающий на вход фильтра. Иначе невозможно накопить всю энергию входного
сигнала для формирования пика на выходе фильтра в момент времени $t_0$. Увеличение $t_0$ сверх
величины $\tau + T$ не влияет на величину максимума выходного сигнала, а лишь сдвигает его в
сторону большего запаздывания. Поэтому имеет смысл выбирать $t_0 = \tau + T$. Тогда максимальное
значение выходного сигнала достигается точно в момент окончания входного импульса.

Сигнал $M(t)$ достигает максимального значения в момент $t_0$, поскольку функция корреляции
всегда имеет максимальное значение в нуле $max(\Psi_M(\tau))=\Psi(0)$. Тогда максимальное значение с
точностью до постоянного множителя $C_0$ равно энергии сигнала: Формула (35)

Сжатие сигнала (его укорочение) прямо пропорционально базе сигнала. В случае ЛЧМ
сигнала база сигнала регулируется значением девиации частоты.  При увеличении девиации
уменьшается $\tau$ - характерное время выходного сигнала (см формулу 53). При уменьшении
$\tau$ увеличивается характерная ширина спектра выходного сигнала (как следствие из
Фурье-преобразования). Получаем, что при увеличении девиации сигнала увеличивается его база. 



