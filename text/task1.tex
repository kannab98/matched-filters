%!TEX root = ../mfilters.tex
\subsection{Задание 1. Простые и сложные сигналы и их свойства}
В этом задании рассматриваются особенности простых и сложных
сигналов, которые проявляются в поведении спектров сигналов. Следует
проследить за тем, какие существуют закономерности при изменении спектров
в зависимости от изменения временных параметров простых и сложных
сигналов.
Для каждого рассмотренного сигнала $m(t)$ строятся графики реализации
сигнала, амплитудного и фазового спектров, а так же функция корреляции и
спектральная плотность энергии.



\subsubsection{Прямоугольный видеоимпульс}

При длительности импульса $\tau= 10$ мс и  $\tau=20$ мс на лабораторной
установке получены (см. рис. \ref{fig:task1_10} и рис.
\ref{fig:task1_20}) амплитудные, фазовые и энергетические спектры.
Экспериментальные зависимости хорошо согласуются с теоретическими, изложенными
выше в \ref{par:analiticheskie_sootnosheniia}.


\begin{figure}[H]
    \centering
    \includegraphics[width=0.9\linewidth]{imgs/t1s1_10.png}
    \caption{Приборная панель виртуального прибора. Моделируется
    прямоугольный видеоимпульс с длительность $\tau=10$ мс}
    \label{fig:task1_10}
\end{figure}
\begin{figure}[H]
    \centering
    \includegraphics[width=0.9\linewidth]{imgs/t1s1_20.png}
    \caption{Приборная панель виртуального прибора. Моделируется
    прямоугольный видеоимпульс с длительность $\tau=20$ мс}
    \label{fig:task1_20}
\end{figure}


\paragraph{Оценка базы}%
\label{par:otsenka_bazy}


Найдем базу для прямоугольного импульса по следующей формуле: 
\begin{equation}
    B = T \cdot \Delta f,
    \label{eq:p:1}
\end{equation}
где $T$ - эффективная длительность, $\Delta f$ - эффективная ширина полосы спектра
сигнала(в качестве оценки берется половина ширины главного лепестка амплитудного спектра).
\begin{equation}
    B_{10ms} = 10^{-2} \cdot 100 = 1, \quad B_{20ms} = 20 \cdot 10^{-3} \cdot 50 = 1
    \label{eq:}
\end{equation}
База прямоугольного импульса равна единице, что означает что это простой сигнал. Таким образом
справедливо соотношение $\Delta f = \frac{1}{T}$. Действительно, в соответствии с этой зависимостью,
наблюдается сужение амплитудного спектра при увеличении длительности сигнала.

\paragraph{Оценка энергии}%
\label{par:otsenka_energii}


Полная энергия прямоугольного сигнала равна $A^2 \tau^2$  (см.
\ref{sub:theory_priamougol_nyi_impul_s} ). Поскольку, за ширину спектра мы
приняли не бесконечные пределы, а только ширину главного лепестка, то следует
пересчитать энергию. Посчитав численно интеграл для видеоимпульса с шириной
спектра $[-100, 100] \text{ Гц}$, получим, что $90.2 \%$ энергии находится в указанном
диапазоне. 


\begin{table}[H]
    \centering
    \begin{tabular}{|l|l|}
    \hline
     Диапазон & $E$ \\ \hline
     $(-\infty, \infty)$&  0.01 \\ \hline
     $(-100, 100)$&  0.00902 \\ \hline
    \end{tabular}
\end{table}

\subsubsection{Прямоугольный видеоимпульс с гармоническим заполнением}

Изучить амплитудный, фазовый и энергетический спектры. Задать
длительность импульса 10мс и 20мс, амплитуду равной 1 и частоту
заполнения 400Гц, а затем проанализировать зависимости.

\begin{figure}[H]
    \centering
    \includegraphics[width=0.9\linewidth]{imgs/t1s2_10.png}
    \caption{Приборная панель виртуального прибора. Моделируется
    прямоугольный радиоимпульс с длительность $\tau=10$ мс}
    \label{fig:task2_10}
\end{figure}
\begin{figure}[H]
    \centering
    \includegraphics[width=0.9\linewidth]{imgs/t1s2_20.png}
    \caption{Приборная панель виртуального прибора. Моделируется
    прямоугольный радиоимпульс с длительность $\tau=20$ мс}
    \label{fig:task2_20}
\end{figure}

При изменении длительности импульса радиоимпульс ведет себя аналогично
видеоимпульсу, что и предсказывает теория.
\begin{equation}
    B_{10ms} = 10^{-2} \cdot 100 = 1, \quad B_{20ms} = 20 \cdot 10^{-2} \cdot 50 = 1
    \label{eq:}
\end{equation}
Значение базы - единица, означает что радиоимпульс это простой сигнал.

\subsubsection{Линейно-частотный модулированный импульс}
Получить временные реализации ЛЧМ сигнала с параметрами:
\begin{itemize}
    \item длительность 100мс, средняя частота заполнения 1000Гц,
    девиация 500Гц;
    \item длительность 100мс, средняя частота заполнения 1000Гц,
    девиация 1000Гц
    \item амплитуда 1.
\end{itemize}
\begin{figure}[H]
    \centering
    \includegraphics[width=0.9\linewidth]{imgs/t1s3_500.png}
    \caption{500 Гц}
    \label{fig:task3_500}
\end{figure}
\begin{figure}[H]
    \centering
    \includegraphics[width=0.9\linewidth]{imgs/t1s3_1000.png}
    \caption{1000 Гц}
    \label{fig:task3_1000}
\end{figure}

\begin{equation}
    B_{500Hz} = 100 \cdot 10^{-3} \cdot (1260-760) = 50, \quad B_{1000Hz} = 100 \cdot 10^{-3} \cdot (1500-500) = 100
    \label{eq:}
\end{equation}

\textbf{Для ЛЧМ сигнала сравнить протяженность корреляционной
функции с длительностью сигнала. Во сколько раз она меньше
длительности сигнала?}

При длительности ЛЧМ сигнала 100 мс, протяженность функции корреляции составила всего 0.4 мс, что в 250 раз меньше.

\textbf{Для ЛЧМ сигнала оценить диапазон изменения фазовых сдвигов у
гармоник сигнала в пределах полосы амплитудного спектра.
Нарисовать амплитудный спектр в приближенном виде
(аппроксимируя прямоугольником) и посмотреть, какой в этих
пределах фазовый спектр.}

Диапазон изменения фазовых сдвигов в случае девиации 500 Гц составил $\phi \in [0 - 160]$ радиан (см. рис.
\ref{fig:task3_500_phase}), в случае девиации 1000 Гц составил $\phi \in [0 - 260]$ радиан (см. рис.
\ref{fig:task3_1000_phase}).
\begin{figure}[H]
    \centering
    \includegraphics[width=0.5\linewidth]{imgs/t1s3_500_extra.png}
    \caption{Диапазон изменения фазовых сдвигов у гармоник сигнала, девиация 500 Гц}
    \label{fig:task3_500_phase}
\end{figure}

\begin{figure}[H]
    \centering
    \includegraphics[width=0.5\linewidth]{imgs/t1s3_1000_extra.png}
    \caption{Диапазон изменения фазовых сдвигов у гармоник сигнала, девиация 1000 Гц}
    \label{fig:task3_1000_phase}
\end{figure}

\subsubsection{Код Баркера}
Получить реализации для кода Баркера (N=13) при длительности 13мс и
26мс
\begin{figure}[H]
    \centering
    \includegraphics[width=0.9\linewidth]{imgs/t1s4_13.png}
    \caption{13 мс}
    \label{fig:task4_13}
\end{figure}

\begin{figure}[H]
    \centering
    \includegraphics[width=0.9\linewidth]{imgs/t1s4_26.png}
    \caption{26 мс}
    \label{fig:task4_26}
\end{figure}

\begin{equation}
    B_{13ms} = 13 \cdot 10^{-3} \cdot 1 = 13 \cdot 10^{-3}, \quad B_{26ms} = 26 \cdot 10^{-3} \cdot 0.5 = 13 \cdot 10^{-3}
    \label{eq:}
\end{equation}

%\textbf{На что ответить в отчете:}
%\begin{enumerate}
    %\item Получить оценку энергии импульса разными способами по
    %экспериментальным данным. Сравнить результаты с
    %теоретическими.
    %\item \textbf{надо пояснения} Для всех четырех видов сигнала оценить базу,
    %используя формулу
    %$B=T \cdot \Delta f$, где T - эффективная длительность, $\Delta f$ - эффективная
    %ширина полосы спектра сигнала. За оценку ширины следует
    %принять половину расстояния между первыми нулями (ширины
    %главного лепестка).
    %\item Пояснить, как изменяется фазовый спектр сигнала, в том диапазоне
    %частот, где лежит основная энергия сигнала. Показать с помощью
    %рисунка, как происходит сложение гармонических составляющих
    %сигнала. Выделить на графиках амплитудного и энергетического
    %спектров диапазон частот, в котором лежит основная энергия
    %сигнала. Как изменяется фазовый спектр сигнала в этом диапазоне
    %частот? Почему физический амплитудный спектр имеет смысл
    %рассматривать только внутри этой полосы?
    %\item \textbf{Done, надо пояснения} Для ЛЧМ сигнала сравнить протяженность корреляционной
    %функции с длительностью сигнала. Во сколько раз она меньше
    %длительности сигнала?
    %\item \textbf{Done, надо пояснения} Для ЛЧМ сигнала оценить диапазон изменения фазовых сдвигов у
    %гармоник сигнала в пределах полосы амплитудного спектра.
    %Нарисовать амплитудный спектр в приближенном виде
    %(аппроксимируя прямоугольником) и посмотреть, какой в этих
    %пределах фазовый спектр.
    %\item Во всех примерах рассматривались изменения спектральных
    %характеристик при изменении временных зависимостей сигналов.
    %Учитывая, что для функций, сопряженных по Фурье, справедливы
    %следующие соотношения (см. Приложение)...(см методичку)
    %\item Чем определяется максимальное значение функции корреляции?
    %Рассмотреть корреляционную функцию как сигнал и найти его базу.
    %\item Сравнить изменения спектрально-корреляционных характеристик
    %при изменении длительности различных сигналов.
%\end{enumerate}

