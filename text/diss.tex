\documentclass[a4paper,14pt]{extarticle}

\usepackage[utf8x]{inputenc}
\usepackage[T2A]{fontenc}
\usepackage[russian]{babel}

\usepackage[mode=buildnew]{standalone}
\usepackage{setspace}


\usepackage
    {
        % Дополнения Американского математического общества (AMS)
        amssymb,
        amsfonts,
        amsmath,
        amsthm,
        % Пакет для физических текстов
        physics,
        % Графики и рисунки
        graphicx,
        subcaption,
        float,
        pgfplots,
        pgfplotstable,
        %caption,
        color,
        booktabs,
        geometry,
        % 
        % Таблицы, списки
        %makecell,
        %multirow,
        %indentfirst,
        %
        % Интегралы и прочие обозначения
        %ulem,
        %esint,
        %esdiff,
        % 
        % Колонтитулы
        fancyhdr,
        pgffor,
    } 

\usepackage{mathtools}
\mathtoolsset{showonlyrefs=true} 

\usepackage{hyperref}
 % Цвета для гиперссылок
\definecolor{linkcolor}{HTML}{000000} % цвет ссылок
\definecolor{urlcolor}{HTML}{799B03} % цвет гиперссылок
 
\hypersetup{linkcolor=linkcolor,urlcolor=urlcolor, colorlinks=true,
citecolor=linkcolor}
\hypersetup{pageanchor=false}
% Увеличенный межстрочный интервал, французские пробелы
\linespread{1.2} 
\frenchspacing 

\newcommand{\mean}[1]{\langle#1\rangle}
\newcommand\ct[1]{\text{\rmfamily\upshape #1}}
\newcommand*{\const}{\ct{const}}
\renewcommand{\phi}{\varphi}
\renewcommand{\epsilon}{\varepsilon}
%\renewcommand{\sigma}{\varsigma}

\usepackage{array}
\usepackage{pstool}

\geometry       
    {
        left            =   2cm,
        right           =   2cm,
        top             =   2.5cm,
        bottom          =   2.5cm,
        bindingoffset   =   0cm
    }

\renewcommand{\contentsname}{Оглавление}
\usepackage{tocloft}
\usepackage{secdot}
\sectiondot{subsection}
