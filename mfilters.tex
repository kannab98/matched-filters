% Тип документа
\documentclass[a4paper,12pt]{extarticle}

% Шрифты, кодировки, символьные таблицы, переносы
\usepackage{cmap}
\usepackage[T2A]{fontenc}
\usepackage[utf8x]{inputenc}
\usepackage[russian]{babel}

% Это пакет -- хитрый пакет, он нужен но не нужен
\usepackage[mode=buildnew]{standalone}

\usepackage
	{
		% Дополнения Американского математического общества (AMS)
		amssymb,
		amsfonts,
		amsmath,
		amsthm,
		physics,
		% misccorr,
		% 
		% Графики и рисунки
		wrapfig,
		graphicx,
		subcaption,
		float,
		tikz,
		tikz-3dplot,
		caption,
		csvsimple,
		color,
		booktabs,
		pgfplots,
		pgfplotstable,
		geometry,
		% 
		% Таблицы, списки
		array,
		makecell,
		multirow,
		indentfirst,
		%
		% Интегралы и прочие обозначения
		ulem,
		esint,
		esdiff,
		% 
		% Колонтитулы
		fancyhdr,
	}  

\usepackage{xcolor}
\usepackage{hyperref}

 % Цвета для гиперссылок
\definecolor{linkcolor}{HTML}{000000} % цвет ссылок
\definecolor{urlcolor}{HTML}{799B03} % цвет гиперссылок
 
\hypersetup{pdfstartview=FitH,  linkcolor=linkcolor,urlcolor=urlcolor, colorlinks=true}
% Обводка текста в TikZ
\usepackage[outline]{contour}

% Увеличенный межстрочный интервал, французские пробелы
\linespread{1.3} 
\frenchspacing 

 
\usetikzlibrary
	{
		decorations.pathreplacing,
		decorations.pathmorphing,
		patterns,
		calc,
		scopes,
		arrows,
		fadings,
		through,
		shapes.misc,
		arrows.meta,
		3d,
		quotes,
		angles,
		babel
	}


\tikzset{
	force/.style=	{
		>=latex,
		draw=blue,
		fill=blue,
				 	}, 
	%				 	
	axis/.style=	{
		densely dashed,
		blue,
		line width=1pt,
		font=\small,
					},
	%
	th/.style=	{
		line width=1pt},
	%
	acceleration/.style={
		>=open triangle 60,
		draw=magenta,
		fill=magenta,
					},
	%
	inforce/.style=	{
		force,
		double equal sign distance=2pt,
					},
	%
	interface/.style={
		pattern = north east lines, 
		draw    = none, 
		pattern color=gray!60,
					},
	cross/.style=	{
		cross out, 
		draw=black, 
		minimum size=2*(#1-\pgflinewidth), 
		inner sep=0pt, outer sep=0pt,
					},
	%
	cargo/.style=	{
		rectangle, 
		fill=black!70, 
		inner sep=2.5mm,
					},
	%
	caption/.style= {
		midway,
		fill=white!20, 
		opacity=0.9
					},
	%
	}

\newenvironment{tikzpict}
    {
	    \begin{figure}[htbp]
		\centering
		\begin{tikzpicture}
    }
    { 
		\end{tikzpicture}
		% \caption{caption}
		% \label{fig:label}
		\end{figure}
    }


\newcommand{\vbLabel}[3]{\draw ($(#1,#2)+(0,5pt)$) -- ($(#1,#2)-(0,5pt)$) node[below]{#3}}
\newcommand{\vaLabel}[3]{\draw ($(#1,#2)+(0,5pt)$) node[above]{#3} -- ($(#1,#2)-(0,5pt)$) }

\newcommand{\hrLabel}[3]{\draw ($(#1,#2)+(5pt,0)$) -- ($(#1,#2)-(5pt,0)$) node[right, xshift=1em]{#3}}
\newcommand{\hlLabel}[3]{\draw ($(#1,#2)+(5pt,0)$) node[left, xshift=-1em]{#3} -- ($(#1,#2)-(5pt,0)$) }



\newcommand\zi{^{\,*}_i}
\newcommand\sumn{\sum_{i=1}^{N}}

\tikzset{
	coordsys/.style={scale=1.8,x={(1.1cm,-0cm)},y={(0.5cm,1cm)}, z={(0cm,0.8cm)}},
	coordsys/.style={scale=1.5,x={(0cm,0cm)},y={(1cm,0cm)}, z={(0cm,1cm)}}, 
	coordsys/.style={scale=1.5,x={(1cm,0cm)},y={(0cm,1cm)}, z={(0cm,0cm)}}, 
}

\usepgfplotslibrary{units}


% Draw line annotation
% Input:
%   #1 Line offset (optional)
%   #2 Line angle
%   #3 Line length
%   #5 Line label
% Example:
%   \lineann[1]{30}{2}{$L_1$}

\newcommand{\lineann}[4][0.5]{%
    \begin{scope}[rotate=#2, blue,inner sep=2pt, ]
        \draw[dashed, blue!40] (0,0) -- +(0,#1)
            node [coordinate, near end] (a) {};
        \draw[dashed, blue!40] (#3,0) -- +(0,#1)
            node [coordinate, near end] (b) {};
        \draw[|<->|] (a) -- node[fill=white, scale=0.8] {#4} (b);
    \end{scope}
}

\newcommand{\lineannn}[4][0.5]{%
    \begin{scope}[rotate=#2, blue,inner sep=2pt, ]
        \draw[dashed, blue!40] (0,0) -- +(0,#1)
            node [coordinate, near end] (a) {};
        \draw[dashed, blue!40] (#3,0) -- +(0,#1)
            node [coordinate, near end] (b) {};
        % \draw[color=white, color=blue] (a) -- node[fill=white, scale=0.8] {#4} (b);
        \draw[->|] (a)++(-0.3,0) -- (a);
        \draw[->|] (b)++(0.3,0) coordinate (xx) -- (b);
        \draw (xx) node[fill=white, scale=0.8, right] {#4};
    \end{scope}
}

% Круговая стрелка относительно центра (дуга из центра)
\tikzset{
  pics/carc/.style args={#1:#2:#3}{
    code={
      \draw[pic actions] (#1:#3) arc(#1:#2:#3);
    }
  },
  dash/.style={
  	dash pattern=on 5mm off 5mm
  }
}



\pgfplotsset{
    % most recent feature set of pgfplots
    compat=newest,
}

% const прямым шрифтом
\newcommand\ct[1]{\text{\rmfamily\upshape #1}}
\newcommand*{\const}{\ct{const}}


\usepackage[europeanresistors,americaninductors]{circuitikz}

% Style to select only points from #1 to #2 (inclusive)
\pgfplotsset{select/.style 2 args={
    x filter/.code={
        \ifnum\coordindex<#1\def\pgfmathresult{}\fi
        \ifnum\coordindex>#2\def\pgfmathresult{}\fi
    }
}}


\usepackage{array}
\usepackage{pstool}



\geometry		
	{
		left			=	2cm,
		right 			=	2cm,
		top 			=	3cm,
		bottom 			=	3cm,
		bindingoffset	=	0cm
	}

%%%%%%%%%%%%%%%%%%%%%%%%%%%%%%%%%%%%%%%%%%%%%%%%%%%%%%%%%%%%%%%%%%%%%%%%%%%%%%%

\fancyfoot{} 

\fancyfoot[C]{\thepage} 

%%%%%%%%%%%%%%%%%%%%%%%%%%%%%%%%%%%%%%%%%%%%%%%%%%%%%%%%%%%%%%%%%%%%%%%%%%%%%%%

\renewcommand{\contentsname}{Оглавление}

\usepackage{tocloft}
\renewcommand{\cftsecleader}{\cftdotfill{\cftdotsep}}
\usepackage{secdot}
\sectiondot{subsection}

\begin{document}

\def\labauthors{Понур К.А., Хавьер, Шиков А.П.}
\def\labgroup{450}
\def\labnumber{1}
\def\labtheme{Согласованные фильтры}
\begin{titlepage}

\begin{center}

{\small\textsc{Нижегородский государственный университет имени Н.\,И. Лобачевского}}
\vskip 1pt \hrule \vskip 3pt
{\small\textsc{Радиофизический факультет. Кафедра статистической радиофизики и мобильных систем связи.}}

\vfill

{\Large Отчет по лабораторной работе №\labnumber\vskip 12pt\bfseries \labtheme}
	
\end{center}

\vfill
	
\begin{flushright}
	{Выполнили студенты \labgroup\ группы\\ \labauthors}%\vskip 12pt Принял:\\ Менсов С.\,Н.}
\end{flushright}
	
\vfill
	
\begin{center}
	Нижний Новгород, \the\year
\end{center}

\end{titlepage}



\newpage

\tableofcontents

\section{Теоретическая часть}
\subsection{Прямоугольный импульс}%
\label{sub:theory_priamougol_nyi_impul_s}



\newcommand{\sinc}[1]{\frac{\sin{#1}}{#1}}
Представим видеоимпульс кусочной функцией:
\begin{equation}
    f(t) = 
    \begin{cases}
        A, &\text{ при } 0 \leq t \leq \tau \\
        0, &\text{ при } t < 0 \text{ и } t > \tau
    \end{cases}
\end{equation}
Получим аналитическое выражения для амплитудного, фазового и
энергетического спектра прямоугольного видеоимпульса. 

Запишем преобразование Фурье от сигнала $f(t)$ 
\begin{equation}
    F(j\omega) = \int\limits_{-\infty}^{\infty} f(t) e^{-j \omega t}  \dd t
    = 
    \frac{A}{-j \omega} e^{-j \omega t}\eval_{0}^{\tau} = 
    A \tau \cdot \sinc{\omega \frac{\tau}{2}} e^{-j \omega \frac{\tau}{2}}
\end{equation}



Амплитудный спектр сигнала $f(t)$ будет модулем преобразования Фурье$\abs{F(j\omega)}$
 \begin{equation}
     S(\omega) = \abs{F(j\omega)} = A \tau \sinc{\omega\frac{\tau}{2}} 
\end{equation}


Фазовый спектр будет равен 
\begin{equation}
    \Psi(\omega) = \frac{\omega\tau}{2}
\end{equation}

Функция корреляции будет равна
\begin{equation}
    K(t) = \int\limits_{-\infty}^{\infty} f(t) f(t+t_{0}) \dd t_0 = 
    \begin{cases}
        A^2 \tau (1 - \frac{\abs{t}}{\tau}), &\text{ при } \abs{t} \leq \tau, \\
        0, &\text{ при } \abs{t} > \tau
    \end{cases}
\end{equation}

Спектральную плотность мощности тоже несложно найти как прямое преобразование
Фурье от функции корреляции $K(t)$:
\begin{equation}
    \begin{gathered}
        E(\omega) = \int\limits_{-\infty}^{\infty} K(t) e^{-j\omega t} \dd t =
        2 A^2 \tau \int\limits_{0}^{\tau}  \qty(1- \frac{t}{\tau} )e^{-j \omega t}\dd t
    \end{gathered}
\end{equation}

\begin{equation}
    \begin{gathered}
        - \frac{1}{\tau}\int\limits_{0}^{\tau}  t e^{-j \omega t}\dd t = 
        \frac{e^{- j \omega \tau} \qty( - j \omega \tau - 1) + 1}{\tau\omega^2}
        \\
        \int\limits_{0}^{\tau}  e^{-j \omega t}\dd t = 
        \frac{1}{-j \omega} \qty(e^{-j\omega\tau} - 1)
    \end{gathered}
\end{equation}





\begin{equation}
    \label{eq:pdf}
    E(\omega) = \qty(A\tau \sinc{\omega \frac{\tau}{2}})^2
\end{equation}

Можем оценить расстояние между нулями главного лепестка энергетического спектра:
\begin{equation}
    \label{eq:DeltaW}
    \Delta \omega = \frac{4 \pi}{\tau}
\end{equation}


Также можно оценить энергию сигнала на бесконечном интервале:
 \begin{equation}
     \label{eq:energy_spec}
     E_0 = \frac{(A \tau)^2}{2\pi}
     \int\limits_{-\infty}^{\infty} \qty(\sinc{\omega \frac{\tau}{2}})^2 \dd
     \omega  = \frac{A^2 \tau^2}{\pi} \int\limits_{-\infty}^{\infty}
     \qty(\sinc{\xi})^2 \dd \xi 
\end{equation}



Интеграл в выражении \eqref{eq:energy_spec} равен $\pi$, а значит энергия на
бесконечном интервале: 
 \begin{equation}
     E_0 = A^2 \tau^2
\end{equation}


\subsection{Прямоугольный импульс с гармоническим заполнением}%
\label{sub:priamougol_nyi_impul_s_s_garmonicheskim_zapolneniem}


Представим радиоимпульс кусочной функцией ($\omega_0 \gg \frac{2\pi}{\tau}$):
\begin{equation}
    f(t) = 
    \begin{cases}
        A\cos (\omega_0 t + \phi_0), &\text{ при } 0 \leq t \leq \tau \\
        0, &\text{ при } t < 0 \text{ и } t > \tau
    \end{cases}
\end{equation}
Получим аналитическое выражения для амплитудного, фазового и
энергетического спектра прямоугольного радиоимпульса. 

Запишем преобразование Фурье от сигнала $f(t)$ 
\begin{equation}
    \begin{gathered}
    F(j\omega) = \int\limits_{-\infty}^{\infty} f(t) e^{-j \omega t}  \dd t =
    \begin{cases}
    \frac{A\tau}{2}  \sinc{ \frac{\omega + \omega_0}{2} \tau} 
    \exp{j \frac{\omega- \omega_0}{2} \tau + j \phi_0} , &\text{ }\omega \geq 0 \\
    \frac{A\tau}{2}  \sinc{ \frac{\omega - \omega_0}{2} \tau} 
    \exp{j \frac{\omega + \omega_0}{2} \tau + j \phi_0} , &\text{  } \omega < 0
    \end{cases}  
    \end{gathered}
\end{equation}




Амплитудный спектр сигнала $f(t)$ будет модулем преобразования Фурье 
и будет вычисляться аналогично случаю прямоугольного импульса
 \begin{equation}
     S(\omega) = \abs{F(j\omega)} 
 \end{equation}

Фазовый спектр по определению будет равен 
\begin{equation}
    \Psi(\omega) = \arctg{ \frac {\Im{F(j \omega)}}{\Re{F(j\omega)}}}
\end{equation}

Функция корреляции будет равна
\begin{equation}
    K(t) = \int\limits_{-\infty}^{\infty} f(t) f(t+t_{0}) \dd t_0 = 
    \begin{cases}
        A^2 \tau (1 - \frac{\abs{t}}{\tau})\cos \omega_0 t, &\text{ при } \abs{t} \leq \tau, \\
        0, &\text{ при } \abs{t} > \tau
    \end{cases}
\end{equation}

Спектральную плотность мощности тоже несложно найти как прямое преобразование
Фурье от функции корреляции $K(t)$:
\begin{equation}
    \begin{gathered}
        E(\omega) = \int\limits_{-\infty}^{\infty} K(t) e^{-j\omega t} \dd t =
        S^2(\omega) + S^2(- \omega)
    \end{gathered}
\end{equation}


При этом энергетика радиоимпульса равно энергетике видеоимпульса
\begin{equation}
    \label{eq:}
    E_0 = A^2 \tau^2
\end{equation}


\newpage
\section{Практическая часть}
Тут практика

\makeatletter
\@for\i:={1,2,3,4,5,6}\do{\input{text/task\i} \newpage}
\makeatother


\section{Вывод}


\newpage
\section{Дополнение}
\textit{Здесь приведены некоторые вопросы, которые разбирались на сдаче отчета}

\end{document}
